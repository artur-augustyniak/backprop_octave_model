\subsection{subset}

\textit{subset} can be used to optimize the data sets for train, test and validation of a neural
network.\\

\noindent \textbf{\textcolor{brown}{Syntax:}}\\

\noindent [mTrain, mTest, mVali] = subset(mData,nTargets,iOpti,fTest,fVali);\\

\noindent \textbf{\textcolor{brown}{Description:}}\\

\noindent  \textbf{Left-Hand-Side:}\\
\noindent mTrain: (R+T) x M matrix with R input rows, T output rows and M columns
				where M <= N.\\
\noindent mTest:  (R+T) x S matrix with R input rows, T output rows and S columns
				where S <= N.\\
\noindent mVali:  (R+T) x U matrix with R input rows, T output rows and U columns
				where U <= N. And U can only exist, if S also exist.\\

\noindent  \textbf{Right-Hand-Side:}\\
\noindent mData: (R+T) x N matrix with R input rows, T output rows and N columns\\ 
\noindent nTargets: Number of T output rows\\ 
\noindent iOpti: Integer value to define level of optimization.\\
\noindent fTest: Fraction to define the percentage of data sets which should be used for testing. \\
\noindent fVali: Fraction to define the percentage of data sets which should be used for testing.\\

\noindent iOpti can have following values:\\
0	: no optimization\\
1	: will randomise the column order and rerange the columns containing min and max values to be in the train set\\
2	:	will NOT randomise the column order, but rerange the columns containing min and max values to be in the train set\\

\noindent fTest or fValie have following meaning:\\
Each of this arguments can be a fraction or zero. The value 1 is not allowed! The sum of both values
must also be smaller than 1!\\
Example: fTest = 1/3\\

\noindent \textbf{Default values}\\
\noindent iOpti		= 1\\
\noindent fTest		= 1/3\\
\noindent fVali		= 1/6\\

\noindent \textbf{\textcolor{brown}{Examples:}}\\

\noindent mTrain = subset(mData,2,1,0,0)\\
\noindent [mTrain,mTest] = subset(mData,2,1,1/3,0);\\
\noindent [mTrain,mTest,mVali] = subset(mData,1);\\
\noindent [mTrain,mTest,mVali] = subset(mData,1,1,1/3,1/6);\\